\section{Análisis de Resultados}

    \subsubsection{Evaluación de la Generación de Terreno}
    % En esta subsección se analizarán los resultados obtenidos en cuanto a la generación de terreno, evaluando su realismo, diversidad y coherencia con respecto a los parámetros de entrada.

    \subsubsection{Comparación de Biomas Generados}
    % Se compararán los biomas generados para evaluar su variedad, apariencia y coherencia con las características del terreno circundante.

    \subsubsection{Evaluación de las Poblaciones}
    % En esta subsección se evaluará la distribución y apariencia de las poblaciones generadas sobre el terreno, considerando factores como densidad, variedad y coherencia con el entorno.

    \subsubsection{Desempeño del Sistema en Tiempo Real}
    % Se analizará el desempeño del sistema en términos de velocidad de generación de terreno y respuesta en tiempo real a cambios en los parámetros de entrada.

    \subsubsection{Compatibilidad con Realidad virtual}
    % Se analizará la capacidad del sistema para generar terrenos de manera compatible con entornos de realidad virtual, evaluando la inmersión y la experiencia del usuario.

    \subsubsection{Experiencia del Usuario}
    % Se recopilarán y analizarán las opiniones y feedback de los usuarios sobre la experiencia de interactuar con el sistema de generación de terreno.

    \subsubsection{Limitaciones y Áreas de Mejora}
    % Se identificarán las limitaciones del sistema y se propondrán posibles mejoras o áreas de investigación futuras.

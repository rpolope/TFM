Para el desarrollo de una herramienta de las características descritas se plantean diferentes cuestionas a resolver para cada uno de los apartados que se pretenden implementar en este proyecto.

\section{Problemáticas en la Generación de Terreno}

En la generación de la malla del terreno aparecen varias problemáticas a abordar que describo a continuación:

\begin{itemize}
    \item La creación de chunks de terreno conforme se mueve el jugador.
    \item La coherencia entre distintos chunks, es decir, que no hayan discontinuidades entre chunks.
    \item Los chunks han de almacenarse de manera eficiente para poder acceder a ellos rápidamente cuando se deban volver a mostrar.
    \item Implementación de LOD.
    \item Implementación de la herramienta para modificar los vértices del terreno a posteriori.
    \item Mantener la concordancia entre chunks con diferente nivel de detalle mediante teselación.
    \item Texturización del terreno generado mediante shaders tripalanares.
    \item Adecuación de los nombres para un uso lo más sencillo posible y mejorar la UX.
\end{itemize}

\section{Problemáticas en la Generación de Biomas}

Para la generación de biomas se han de tener en cuenta los siguientes aspectos:
\begin{itemize}
    \item Manejo de los algoritmos de generación para usar aquellos que produzcan resultados que mejor se ajusten a cada tipo de bioma por su orografía característica.
    \item La diferenciación de texturas entre mallas de terreno pertenecientes a diferentes biomas contiguos para que haya discontinuidades muy notables.
    \item Los cambios en la parametrización para ajustar la orografía del terreno a su bioma. 
    \item La generación de assets propios para cada bioma y su distribución, como rocas o vegetación. 
    \item Generación de flora y fauna (agentes autónomos) coherente (en sitios correctos y con una distribución correcta).
    % \item Simulación de fenómenos climáticos y físicos como ciclos de día y noche, niebla, lluvias o ventiscas y tormentas de arena en momentos aleatorios.
    \item Generación de sonidos para cada tipo de biomas con transiciones suaves.
    \item Ajuste mediante teselación entre mallas de chunks pertenecientes a diferentes biomas.
    \item Presencia de cuerpos de agua como lagos, mares o ríos han de ser considerados a la hora de generar los terrenos de cada bioma.
    \item Instanciación eficiente de los assets mencionados mediante el uso de estructuras de datos o algoritmos específicos.
\end{itemize}

\section{Problemáticas en la Generación de Poblaciones}

En cuanto a la generación de poblaciones se han considerado que hay que prestar atención a los siguientes factores:
\begin{itemize}
    \item La distribución de edificios ha de ser coherente y regida por ciertas reglas.
    \item Debe haber un diversidad visual entre los edificios, por lo que se deberán distribuir las diferentes constricciones de manera que cree un resultado realista.
    \item Las poblaciones deben ser coherentes con el entrono donde se implanten.
\end{itemize}

\section{Problemáticas en la Compatibilidad con Realidad Virtual}

Por último, para la compatibilidad con RV, se necesitará integrar las siguientes funcionalidades.
\begin{itemize}
    \item Uso de DOTS de Unity para paralelización del código.
    \item Prevención de los \textit{'memory leaks'} para su rendimiento óptimo en tiempo de ejecución, etc.
    \item Optimización del cálculo de colisiones y físicas para un mejor rendimiento.
    \item Optimización del ambiente inmersivo para una mejor experiencia de juego.
    \item Ajuste de la escala de los elementos.
\end{itemize}

% \newpage
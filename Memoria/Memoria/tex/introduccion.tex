% Contenidos del capítulo.
% Las secciones presentadas son orientativas y no representan
% necesariamente la organización que debe tener este capítulo.

\section{Introducción}

En este apartado haré una contextualización del tema de este TFM en la industria de los videojuegos actualmente y daré una visión general de la estructura del proyecto, indicando brevemente de qué partes se compondrá y cuál será el proceso de creación del mismo.

Durante los últimos años, y más concretamente durante la última década, el desarrollo de la industria de los videojuegos ha sido testigo de una evolución enorme, dado el auge del consumo de videojuegos y todo tipo de productos multimedia y digitales.

Esta tendencia, que se ha agudizado en los últimos años aún más, ha provocado que las experiencias que ofrecen los videojuegos sean cada vez más completas, más cautivadoras e inmersivas, dando lugar a desafíos técnicos cada vez mayores. Como consecuencia de estos cambios, se han producido adaptaciones desde el punto de vista técnico por parte de los desarrolladores quienes han tenido que buscar nuevas técnicas y métodos para generar contenido que cumpla las expectativas y requisitos de la audiencia y consumidores, y no sólo esto, los avances técnicos provocan una carrera y competencia entre los distintos desarrolladores y compañías por crear los mejores resultados y esto ha llevado la innovación a un nivel aún mayor.

Dentro de las nuevas técnicas y métodos que se han implementado en el desarrollo de videojuegos en los últimos años entra la Generación de terreno procedural, tema sobre el que trata esta TFM, y en el que se ha intentado realizar una herramienta con la que obtener resultados de gran calidad de manera eficiente y convincente. 

Uno de los motivos por el que la generación procedural de terreno a ganado popularidad en este auge en la generación de videojuegos ha sido la popularización de los juegos de mundo abierto. Generar terrenos continuos y sin interrupciones en los límites de dichos mundos notables ha generado experiencias que han sido recompensadas por la audiencia. Juegos como No Man's Sky o Diablo son prueba del reconocimiento que han logrado algunos títulos que han usado esta técnica. Pero ante todo, Minecraft es el juego que sin duda ocupa el primer puesto en el uso de la generación procedural, siendo una referencia gracias a su éxito cosechado en multitud de juegos que pretenden emplear esta técnica.

Parte del éxito de este tipo de juegos es la renovación del terreno de juego, creando experiencias diferentes cada vez de manera ilimitado, lo que puede suponer una fuente de entretenimiento en sí mismo. Si además estos espacios de juego generados, son medianamente realistas, cuentan con diversidad de paisajes, climas, simulaciones físicas y demás componentes que contribuyen a crear una sensación inmersiva, se está dando lugar a un activo diferencial en la jugabilidad y en la competitividad del juego frente a los demás. Por otro lado, la influencia de la generación procedural de terrenos no se limita únicamente a la creación de entornos y espacios. También se destaca en otros aspectos de la industria de los videojuegos como la generación de ciudades completas en juegos de mundo abierto, la creación de misiones y contenido diverso que aumenta la rejugabilidad o la generación de personajes. La música y los efectos de sonido también se benefician de la generación procedural, adaptando la banda sonora y la atmósfera del juego en tiempo real para acompañar la acción.

Por las razones mencionadas es que la generación procedural se ha convertido en un activo muy importante en la última generación de videojuegos, pero las razones mencionadas no son la únicas que han contribuido a ello. Además de la generación de la malla del terreno, la generación de texturas y elementos artísticos, suponen un ahorro de tiempo significativo de tiempo y espacio de almacenamiento, lo que colateralmente permite una mayor inversión en otros campos del desarrollo que ya no tienen que ser destinada a la generación del 'playground'. 

La estructura de este proyecto organiza en un establecimiento de los objetivos a alcanzar, que en este caso son la generación del propio terreno, la generación de biomas (con sonidos, vegetación y cambios en el aspecto del terreno y orografía), la generación de espacios habitados, como aldeas o ciudades y su compatibilidad con realidad virtual, por lo que un rendimiento óptimo es fundamental para poder alcanzar los objetivos mencionados. Para ello se hará uso del sistema DOTS de Unity tratando de aprovechar lo más posible la capacidad de la CPU del equipo y permitir la generación en tiempo real y compatible con RV. También se hará una investigación y análisis previos de las tecnologías, técnicas y aplicaciones ya existentes que tienen un fin similar al de este proyecto para hacer uso de ellas en el diseño y la implementación de este TFM. En cuanto al análisis, cabe mencionar que se hará un análisis temporal y económico del coste del proyecto. 

Por último se llevará a cabo la implementación realizando un diseño UML a partir del cual producir el código. Se realizarán pruebas de rendimiento y se obtendrán conclusiones de ellas. Además, se realizarán valoraciones sobre futuros trabajos o ampliaciones de este proyecto.

Cabe mencionar para finalizar que además se implementará un pequeño gameplay a modo de demostración de cómo funciona el resultado obtenido en el ámbito de la gamificación. Para esto se implementará un agente que navegará por el terreno mostrando los diferentes aspectos que se desarrollan en este TFM, como la eficiencia en la generación, los biomas y poblados, la distribución de los elementos que se instancian, la vegetación, etc.
\section{Motivación}

Como ya he mencionado en la introducción, la industria del videojuego exige cada vez más productos más convincentes que satisfagan la demanda de los consumidores. Dentro del industria de los videojuegos existen diferentes categorías de productoras de videojuegos dependiendo de su tamaño y de la calidad técnica y visual de sus productos. La principal motivación de este proyecto es la creación de un asset que permita a los estudios con menos recursos poder realizar contenido de mundo abierto y de calidad, con una atmósfera inmersiva con la que crear mundos que transmitan experiencias, que permitan una buena jugabilidad y con los que poder llevar a cabo los trabajos en la imaginación de los desarrolladores, que junto con el desarrollo de otras tecnologías como la IA en combinación, se puedan lograr trabajos de gran calidad para dar alcance a un mayor número de gente a producir los resultados que deseen. Además de esto, también supone un reto desde el punto de vista técnico conseguir incluir todas estas funcionalidades y características en una herramienta y hacerlo de tal manera que sirva para ser compatible con tecnologías en auge como la RV, con la cual se alinea en la búsqueda de experiencias inmersivas. 

La búsqueda de soluciones técnicas innovadoras y la aspiración de contribuir al crecimiento de la comunidad de desarrolladores son las motivaciones que guían este esfuerzo. La generación procedural de terrenos en Unity tiene el potencial de transformar la forma en que se crean mundos virtuales, y este proyecto se esfuerza por lograr precisamente eso.
\newpage

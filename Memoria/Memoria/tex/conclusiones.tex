% Contenidos del capítulo.
% Las secciones presentadas son orientativas y no representan
% necesariamente la organización que debe tener este capítulo.

\section{Conclusiones}

\subsection{Logros y Contribuciones}
% En esta subsección se resumirán los principales logros y contribuciones del proyecto, destacando cómo ha avanzado el campo de la generación de terreno y cuál es el aporte específico de este trabajo.

\subsection{Cumplimiento de Objetivos}
% Se evaluará en qué medida se lograron los objetivos establecidos al inicio del proyecto, discutiendo los resultados obtenidos y su relación con los objetivos planteados.

\subsection{Lecciones Aprendidas}
% Se compartirán las lecciones aprendidas durante el desarrollo del proyecto, incluyendo tanto aspectos técnicos como de gestión y organización.

\subsection{Trabajo Futuro}
% Se propondrán posibles direcciones para futuras investigaciones o desarrollos relacionados con el tema de la generación de terreno, identificando áreas de mejora y posibles extensiones del trabajo realizado.

\subsection{Conclusión General}
% Se presentará una conclusión general que sintetice los resultados obtenidos, destaque la importancia del trabajo y sugiera posibles implicaciones y aplicaciones futuras.


Este es el prefacio de la memoria. A continuación, se presenta una breve descripción de los capítulos que componen esta memoria:

\begin{description}
  \item[Capítulo 1. INTRODUCCIÓN.] En la introducción se hace una introducción a la generación procedural y su influencia en la industria de los videojuegos, destacando su evolución y su impacto. También se destacan las motivaciones que han impulsado este proyecto y algunas de los objetivos que se pretenden llevar a cabo con esta propuesta. 
  
  \item[Capítulo 2. PLANTEAMIENTO DEL PROBLEMA.] Como se indica en el propio nombre, se indican los obstáculos que plantea la generación procedural, la distinción de biomas, optimización temporal, la continuidad del terreno, distribución de poblaciones y objetos, así como el realismo de los resultados.
  
  \item[Capítulo 3. OBJETIVOS.] Se indican los objetivos que se pretenden alcanzar en el proyecto y cómo se pretende conseguirlos haciendo una diferenciación entre objetivos generales y parciales.

  \item[Capítulo 4. ESTADO DEL ARTE.] Aquí se expondrá el estado del arte, realizando un análisis de las soluciones para generación procedural, estudiando herramientas ya existentes en el mercado con objetivos similares a la propuesta de este proyecto, se analizan métodos y técnicas relacionadas con la generación de terrenos, biomas y ciudades que abordan los temas más principales y secundarios, así como técnicas para la optimización y mejora del rendimiento general, además de factores claves para la compatibilidad con la realidad virtual. También se sacarán conclusiones de la investigación realizada.

  \item[Capítulo 5. DESARROLLO DE LA SOLUCIÓN.] En este capítulo se detalla el análisis, diseño e implementación de la todos los componentes, sistemas y algoritmos empleados para la implementación en Unity de la propuesta de este TFM.

  \item[Capítulo 6. ANÁLISIS DE RESULTADOS.] Finalmente se recogen todos los resultados de las pruebas establecidas para la evaluación de los algoritmos que crean los terrenos de la herramienta. Se describe el conjunto de parámetros empleados y las métricas establecidas para la evaluación de la variable de interés y para la validación de los resultados, tanto visuales como de rendimiento. Se analizan los resultados obtenidos y se identifican los factores y limitaciones que han podido influir en ellos.

  \item[Capítulo 7. CONCLUSIONES Y LÍNEAS FUTURAS.] Por último, se extraen las conclusiones a partir del diseño, desarrollo y experimentación del método propuesto para la elaboración de la herramienta. Se describen futuras líneas de trabajo que complementen el desarrollo alcanzado y se indica la aplicación de este proyecto en el ámbito del desarrollo de contenido multimedia.
\end{description}
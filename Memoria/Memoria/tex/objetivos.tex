% Contenidos del capítulo.
% Las secciones presentadas son orientativas y no representan
% necesariamente la organización que debe tener este capítulo.


\section{Objetivos Generales y Específicos}

\subsection{Objetivos Generales}
El objetivo general de este proyecto es desarrollar una herramienta de generación procedural de terreno en Unity que permita a los desarrolladores de videojuegos crear mundos convincentes e inmersivos, con una alta calidad y realismo, dinámicos y que sea compatible con la realidad virtual (RV).

\subsection{Objetivos Específicos}
Para la consecución del objetivo general se han de conseguir los siguientes objetivos parciales:

\begin{itemize}
   \item Generación de una malla 3D de manera procedural y manejo de esta mediante el uso de varios algoritmos de generación y estructura de datos.
   \item Parametrización mediante interfaz de usuario.
   \item Implementación de herramienta de edición del terreno generado.
   \item Texturización de la misma mediante shaders acorde a su nivel de altura y al bioma que representa.
   \item Instanciación de elementos correspondientes al bioma.
   \item Generación de sonidos, vegetación y poblaciones acordes a los biomas.
   \item Optimización de la generación del terreno, las instanciaciones y cálculo de físicas mediante DOTS de Unity.
   \item Adaptación de la escala del contenido generado y optimización de la inmersividad.
\end{itemize}
\newpage

